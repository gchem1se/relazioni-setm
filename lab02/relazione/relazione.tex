\documentclass{article}
\usepackage[utf8]{inputenc}
\usepackage{xcolor}
\usepackage{siunitx}
\usepackage{amsmath}
\usepackage{float}
\usepackage{graphicx}
\usepackage[italian]{babel}
\usepackage[normalem]{ulem}
\usepackage{circuitikz}

\ctikzset{voltage/bump b/.initial=0}

\renewcommand*\contentsname{Indice}

\title{Caratterizzazione di filtri RC passa-alto e passa-basso}
\author{Simone Aronica, Giovanni Bloise, \\
Gabriele Camisa, Giuseppe Casale}

\begin{document}

\maketitle
\tableofcontents
\pagebreak

\section{Strumenti usati}
\begin{itemize}
    \item Multimetro HP 34401A
    \item Rigol DS1054-Z
    \item Generatore di funzioni
\end{itemize}


\section{Sintesi dell'esperienza}

In questa esperienza di laboratorio è stato osservato il comportamento in frequenza di un circuito stampato che implementa un filtro RC passa-basso e passa-alto.

Come valutazione preliminare è stata verificata la compatibilità del valore effettivo della resistenza integrata nel circuito rispetto a quello nominale dichiarato dal produttore.
È stato trovato che questo valore è di $1021\pm11.2\ \SI{}{\ohm}$, che è compatibile con il valore dichiarato di $1000\pm800\ \SI{}{\ohm}$.
I condensatori sono stati considerati, come dichiarato dal produttore, di capacità $C=\SI{10}{\nano\farad}, 20\%$.

Dunque è stato impostato un generatore di segnali alla frequenza di $\SI{100}{\hertz}$ per generare un segnale sinusoidale di tensione picco-picco $\SI{800}{\milli\volt}$.\\
Tutte le misurazioni sono state effettuate tramite un oscilloscopio, tenendo il coefficiente di sensibilità verticale a $100\frac{\SI{}{\milli\volt}}{\text{div}}$.\\
Per il campionamento della risposta in frequenza del segnale d'ingresso è stato variato da $\SI{200}{\hertz}$ a $\SI{1}{\mega\hertz}$.

\section{Filtro passa-basso}
\begin{center}
\begin{circuitikz}
	\draw (0,0) to[short, *-*] (7,0)
	(3,0) to[C=$\SI{10}{\nano\farad}$, -] (3,2)
	(5,0) to[C=$\SI{10}{\nano\farad}$, -] (5,2)
	(0,2) to[R=$\SI{1}{\kilo\ohm}$, *-] (3,2)
	(3,2) -- (7,2)
	(0,0) to[open, v=$V_{\text{in}}$] (0,2)
	(7,0) to[open, v=$V_{\text{out}}$] (7,2)
	to[short, -*] (7,2);
\end{circuitikz}
\end{center}

\subsection{Frequenza di taglio}
La frequenza di taglio del filtro passa-basso è pari a:
\begin{equation*}
	f_{T}=\frac{1}{2\pi RC}=\frac{1}{2\pi\cdot\SI{1021}{\ohm}\cdot\SI{20}{\nano\farad}}=\SI{8.0}{\kilo\hertz}
\end{equation*}
con un'incertezza pari a:
\begin{equation*}
	\epsilon f_{T}=\epsilon R+\epsilon C = 5\%+40\% = 45\% \Rightarrow\delta f_T = \SI{3.6}{\kilo\hertz}
\end{equation*}

\subsection{Campionamento della risposta in frequenza}

\begin{table}[H]
	\makebox[\linewidth][c]{
		\begin{tabular}{|c|c|c|c|c|}
      % heading
      \hline
			$\text{Freq.\ /\SI{}{\hertz}}$ &
			$V_{\text{in}}\pm\delta V_{\text{in}}\ /\SI{}{\milli\volt}$ &
      $V_{\text{out}}\pm\delta V_{\text{out}}\ /\SI{}{\milli\volt}$ &
			$\Delta\phi\pm\delta(\Delta\phi)\ /\SI{}{\degree}$ &
			$20\log_{10}{(V_{\text{out}} / V_{\text{in}})}\ /\SI{}{\dB}$\\
      \hline
      %%% .1
			100 & $880\pm 80$ & $800\pm80$ & $0.0\pm1.44$ & -0.83\\
			\hline
			300 & $880\pm 80$ & $800\pm80$ & $0.0\pm4.32$ & -0.83\\
			\hline
			500 & $880\pm 80$ & $800\pm80$ & $0.0\pm7.2$ & -0.83\\
			\hline
			1k & $880\pm 80$ & $800\pm80$ & $3.6\pm7.2$ & -0.83\\
			\hline
			3k & $880\pm 80$ & $800\pm80$ & $8.64\pm8.64$ & -0.83\\
			\hline
			5k & $880\pm 80$ & $800\pm80$ & $18.0\pm3.6$ & -0.83\\
			\hline
			10k & $880\pm 80$ & $680\pm80$ & $29.52\pm1.44$ & -2.24\\
			\hline
			30k & $880\pm 80$ & $400\pm80$ & $62.64\pm4.32$ & -6.85\\
			\hline
			50k & $880\pm 80$ & $240\pm80$ & $72.0\pm3.6$ & -11.29\\
			\hline
			100k & $880\pm 80$ & $130\pm8$ & $79.2\pm7.2$ & -16.61\\
			\hline
			300k & $880\pm 80$ & $44\pm4$ & $95.04\pm4.32$ & -26.02\\
			\hline
			500k & $880\pm 80$ & $28\pm4$ & $93.6\pm7.2$ & -29.95\\
			\hline
			1M & $880\pm 80$ & $12\pm4$ & $90.0\pm14.4$ & -37.31\\
			\hline
		\end{tabular}
	}
	\caption{Misurazioni per filtro passa-basso}
\end{table}

\subsection{Diagrammi di Bode}

\begin{figure}[H]
  \makebox[\textwidth][c]{\includegraphics[width=2\linewidth]{img/PASSABASSOMODULO.png}}
  \caption{Diagramma di Bode del modulo per il filtro passa-basso}
\end{figure}
\begin{figure}[H]
  \makebox[\textwidth][c]{\includegraphics[width=2\linewidth]{img/PASSABASSOFASE.png}}
  \caption{Diagramma di Bode della fase per il filtro passa-basso}
\end{figure}

\section{Filtro passa-alto}

\begin{center}
\begin{circuitikz}
	\draw (0,0) to[short, *-*] (5,0)
	(3,0) to[R=$\SI{1}{\kilo\ohm}$, -] (3,2)
	(0,0) to[open, v=$V_{\text{in}}$] (0,2)
	(5,0) to[open, v=$V_{\text{out}}$] (5,2)
	(0,2) to[C=$\SI{10}{\nano\farad}$, *-] (3,2)
	to[short, -*] (5,2);
\end{circuitikz}
\end{center}


\subsection{Frequenza di taglio}
La frequenza di taglio del filtro passa-basso è pari a:
\begin{equation*}
	f_{T}=\frac{1}{2\pi RC}=\frac{1}{2\pi\cdot\SI{1021}{\ohm}\cdot\SI{10}{\nano\farad}}=\SI{16.0}{\kilo\hertz}
\end{equation*}
con un'incertezza pari a:
\begin{equation*}
	\epsilon f_{T}=\epsilon R+\epsilon C = 5\%+20\% = 25\% \Rightarrow\delta f_T = \SI{4.0}{\kilo\hertz}
\end{equation*}

\subsection{Campionamento della risposta in frequenza}

\begin{table}[H]
	\makebox[\textwidth][c]{
		\begin{tabular}{|c|c|c|c|c|}
      % heading
      \hline
			$\text{Freq.\ /\SI{}{\hertz}}$ &
			$V_{\text{in}}\pm\delta V_{\text{in}}\ /\SI{}{\milli\volt}$ &
      $V_{\text{out}}\pm\delta V_{\text{out}}\ /\SI{}{\milli\volt}$ &
			$\Delta\phi\pm\delta(\Delta\phi)\ /\SI{}{\degree}$ &
			$20\log_{10}{(V_{\text{out}} / V_{\text{in}})}\ /\SI{}{\dB}$\\
      \hline
      %%% 
      100 & $880\pm 80$ & $5\pm0.5$ & $93.6\pm18.0$ & -44.91\\
			\hline
			300 & $880\pm 80$ & $16\pm3$ & $86.4\pm10.8$ & -34.81\\
			\hline
			500 & $880\pm 80$ & $16\pm8$ & $45.0\pm18.0$ & -34.81\\
			\hline
			1k & $880\pm 80$ & $80\pm80$ & $90.0\pm9.0$ & -20.83\\
			\hline
			3k & $880\pm 80$ & $160\pm80$ & $86.4\pm4.32$ & -14.81\\
			\hline
			5k & $880\pm 80$ & $240\pm80$ & $72.0\pm7.2$ & -11.29\\
			\hline
			10k & $880\pm 80$ & $400\pm80$ & $57.6\pm3.6$ & -6.85\\
			\hline
			30k & $880\pm 80$ & $720\pm80$ & $27.0\pm2.16$ & -1.74\\
			\hline
			50k & $880\pm 80$ & $760\pm80$ & $18.0\pm3.6$ & -1.27\\
			\hline
			100k & $880\pm 80$ & $760\pm80$ & $9.0\pm3.6$ & -1.27\\
			\hline
			300k & $880\pm 80$ & $800\pm80$ & $4.32\pm4.32$ & -0.83\\
			\hline
			500k & $880\pm 80$ & $800\pm80$ & $0.0\pm3.6$ & -0.83\\
			\hline
			1M & $880\pm 80$ & $800\pm80$ & $0.0\pm7.2$ & -0.83\\
			\hline
		\end{tabular}
	}
	\caption{Misurazioni per filtro passa-alto}
\end{table}

\subsection{Diagrammi di Bode}

\begin{figure}[H]
  \makebox[\textwidth][c]{\includegraphics[width=2\linewidth]{img/PASSAALTOMODULO.png}}
  \caption{Diagramma di Bode del modulo per il filtro passa-alto}
\end{figure}
\begin{figure}[H]
  \makebox[\textwidth][c]{\includegraphics[width=2\linewidth]{img/PASSAALTOFASE.png}}
  \caption{Diagramma di Bode della fase per il filtro passa-alto}
\end{figure}

\section{Conclusioni}
Dai dati raccolti abbiamo ottenuto interpolazioni delle funzioni di trasferimento empiriche 
e ne abbiamo notato la coerenza con i modelli teorici di trasferimento per filtro passa-basso e passa-alto.
In particolare verifichiamo:
\begin{itemize}
	\item l'attenuazione di $\SI{3}{\dB}$ del modulo del segnale in corrispondenza della frequenza di taglio  
	\item lo sfasamento di $\SI{90}{\degree}$ del segnale nello spazio di due decadi centrato nella frequenza di taglio.
\end{itemize}

\end{document}
