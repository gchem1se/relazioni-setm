\documentclass{article}
\usepackage[utf8]{inputenc}
\usepackage{listings}
\usepackage{xcolor}
\usepackage{siunitx}
\usepackage{amsmath}
\usepackage{float}
\usepackage{graphicx}
\usepackage[italian]{babel}
\definecolor{keyword}{rgb}{0,0,1}
\definecolor{codegray}{rgb}{0.5,0.5,0.5}
\definecolor{string}{rgb}{0,0.7,0.2}
\definecolor{backcolour}{rgb}{1,1,1}

\lstdefinestyle{mystyle}{
    backgroundcolor=\color{backcolour},   
    commentstyle=\color{codegray},
    keywordstyle=\color{keyword},
    numberstyle=\tiny\color{codegray},
    stringstyle=\color{string},
    basicstyle=\ttfamily\footnotesize,
    breakatwhitespace=false,         
    breaklines=true,                 
    captionpos=b,                    
    keepspaces=true,                 
    numbers=left,                    
    numbersep=5pt,                  
    showspaces=false,                
    showstringspaces=false,
    showtabs=false,                  
    tabsize=2
}

\lstset{style=mystyle}

\renewcommand*\contentsname{Indice}

\title{Misura di temperatura con
sensore PT100 e
scheda Arduino Uno}
\author{Simone Aronica, Giovanni Bloise, \\
Gabriele Camisa, Giuseppe Casale}

\begin{document}

\maketitle
\tableofcontents
\pagebreak

\section{Strumenti usati}
\begin{itemize}
    \item Arduino Uno, utilizzato come ADC 
    \begin{itemize}
        \item risoluzione di 10 bit 
        \item alimentazione tramite porta USB 2.0 ($V_{\text{CC}} = 5 \pm \SI{0.25}{\volt}$).
    \end{itemize}
    \item Sensore PT100 a film sottile, classe II (B) 
    \begin{itemize}
        \item resistenza variabile secondo $R_{\theta}=R_0\cdot\left(1+A\cdot\theta+B\cdot\theta^2\right)$
        \item parametri caratteristici:
        \begin{itemize}
            \item $R_0=\SI{100}{\ohm}$
            \item $A=3.9083\cdot10^{-3}\SI{}{\per\celsius}$
            \item $B=-5.775\cdot10^{-7}\SI{}{\per\celsius}$
        \end{itemize}
    \item classe II $\rightarrow \delta\theta^\text{sens}=(0.3+0.005\cdot|\theta|)\SI{}{\celsius}$. Il termine incorpora le incertezze dovute ai parametri $A$, $B$ e $R_0$, dei quali non sono disponibili le singole incertezze.
    \item resistenza termica: $\SI{100}{\celsius\per\watt}$
    \end{itemize}
    \item Multimetro HP 34401A
    \begin{itemize}
        \item Risoluzione a 6$\frac{1}{2}$ cifre.
        \item funzioni di misurazione: tensione CC / CA, corrente CC / CA, resistenza a 2 e 4 fili, diodo, continuità, frequenza, periodo.
        \item Ingresso in tensione max 1000 V, ingresso in corrente 3 A max.
    \end{itemize}
\end{itemize}

\section{Sintesi dell'esperienza}
L'esperienza consiste nella misurazione della temperatura ambientale in laboratorio mediante la scheda a microcontrollore Arduino Uno e un sensore di temperatura PT100. Bisogna anche riportare la correlazione di tali misure all'incertezza attesa e a valori di riferimento osservati su un secondo termometro digitale più accurato.

\section{Circuito di condizionamento}
La scheda Arduino Uno implementa un convertitore analogico-digitale a 10 bit con cui è possibile convertire un voltaggio in input, in un range da \SI{0}{\volt} alla tensione di riferimento di \SI{5}{\volt}, in codici digitali da 0 a 1023. 
Il sensore PT100 incorpora una resistenza dipendente dalla temperatura, per cui è possibile ottenere come output la tensione su una resistenza posta in serie al sensore stesso.
Dato che la tensione di alimentazione dell'ADC è usata anche come tensione di riferimento, la misura è definibile raziometrica.

Per ottimizzare la sensibilità $S_{V_F}^{\theta}=\frac{\partial{V_F}}{\partial{\theta}}$, rimanendo contemporaneamente in condizioni di autoriscaldamento accettabili, si è collegato il PT100 in una configurazione di partizione resistiva con un resistore di valore nominale $R_F=\SI{1}{\kilo\ohm}$ (successivamente rivalutato a $(976.00\pm0.11)\SI{}{\ohm}$ tramite il multimetro digitale). La tensione di output del circuito di condizionamento $V_F$ è stata prelevata ai capi di $R_F$.
La serie è stata alimentata tramite USB 2.0 ad una tensione di $V_S=V_\text{CC}=(5.00\pm0.25)\SI{}{\volt}$.
Con questa configurazione si ha una sensibilità di $S_{V_F}^{\theta}=\SI{-1.54}{\milli\volt\per\celsius}$ e un autoriscaldamento $\Delta\theta_{\text{s-o}}\approx\SI{0.2}{\celsius}$.
\subsection{Valutazione dell'incertezza}
La funzione di taratura
\begin{equation*}
    \theta=-\frac{A}{2\cdot B} - \sqrt{\frac{A}{4\cdot B^2} - \frac{1}{R_0\cdot B}\cdot\left(R_0+R_F-\frac{V_S}{V_F}\cdot R_F\right)}
\end{equation*}
si può manipolare dato che si sta operando utilizzando la stessa tensione $V_{\text{CC}}$ di alimentazione come riferimento, ridefinendola come:
\begin{equation*}
    \theta=-\frac{A}{2\cdot B} - \sqrt{\frac{A}{4\cdot B^2} - \frac{1}{R_0\cdot B}\cdot\left(R_0+R_F-\frac{2^{N_b}}{D_\text{out}}\cdot R_F\right)}
\end{equation*}
dato che 
\begin{equation*}
    \begin{split}
        &D_{\text{out}} = V_{\text{CC}} \cdot \frac{R_F}{R_F + R_{\theta}}\cdot\frac{1}{V_{q}}\\
        &V_q = \frac{V_{FR}}{2^{N_b}}=\frac{V_{CC}}{2^{N_b}} \implies \\
        \implies&D_{\text{out}} = V_{\text{CC}} \cdot \frac{R_F}{R_F + R_{\theta}}\cdot\frac{2^{N_b}}{V_{CC}}\\
        &D_{\text{out}} = 2^{N_b} \cdot \frac{R_F}{R_F + R_{\theta}}
    \end{split}
\end{equation*}
L'incertezza assoluta sulla temperatura dipende allora, secondo il modello deterministico, solo da $D_{\text{out}}$, $R_F$ e dalle incertezze sulle caratteristiche del sensore.
\begin{equation*}
    \begin{split}
        \delta\theta&=\left|\frac{\partial{\theta}}{\partial{D_\text{out}}}\right|\delta D_\text{out}+\left|\frac{\partial{\theta}}{\partial{R_\text{F}}}\right|\delta R_\text{F}+\delta \theta^\text{sens}=\\
        &=\left|\frac{2^{N_b-1}\cdot R_F}{B\cdot R_0\cdot D_\text{out}^2\cdot\sqrt{\frac{A^2}{4B^2}-\frac{R_0+R_F-\frac{2^{N_b}\cdot R_F}{D_\text{out}}}{B\cdot R_0}}}\right|\delta D_\text{out}+\\
        &+\left|\frac{1-\frac{2^{N_b}}{D_\text{out}}}{2\cdot B\cdot R_0 \cdot \sqrt{\frac{A^2}{4B^2}-\frac{R_0+R_F-\frac{2^{N_b}\cdot R_F}{D_\text{out}}}{B\cdot R_0}}}\right|\delta R_\text{F}+\\
        &+\delta \theta^\text{sens}
    \end{split}
\end{equation*}


\subsection{Firmware}
La funzione di taratura è stata implementata, insieme alla visualizzazione delle misure di temperatura nel seguente codice sorgente:
\begin{lstlisting}[language=C]
const int pin = A3;
const long Rf = 976;
const long R0 = 100;
const double A = 3.9083*pow(10, -3);
const double B = -5.775*pow(10, -7);
const int Vs = 5;

void setup() {
    pinMode(pin, INPUT);
    Serial.begin(9600);
}

void loop() {
    int Dout = analogRead(pin); 
    double T = -(A/(2*B))-sqrt(pow(A,2)/(4*pow(B,2))-1/(R0*B)*(R0+Rf-pow(2, 10)/Dout*Rf));
    
    Serial.print("Valore sensore: ");
    Serial.println(Dout);
    Serial.print("Valore temperatura in Celsius: ");
    Serial.println(T);
    Serial.println("######");
    
    delay(1000);
}
\end{lstlisting}
\subsection{Dithering}
\begin{table}[H]
    \makebox[\linewidth][c]{
        \begin{tabular}{|c|c|c|}
            \hline
            $D_\text{out}$ & $\theta\ /\SI{}{\celsius}$ & $\delta\theta\ /\SI{}{\celsius}$\\
            \hline
            920 & 26.54 & 6.95\\
            \hline
            921 & 23.50 & 6.91\\
            \hline
            921 & 23.50 & 6.91\\
            \hline
            921 & 23.50 & 6.91\\
            \hline
            921 & 23.50 & 6.91\\
            \hline
            921 & 23.50 & 6.91\\
            \hline
            921 & 23.50 & 6.91\\
            \hline
            921 & 23.50 & 6.91\\
            \hline
            921 & 23.50 & 6.91\\
            \hline
            920 & 26.54 & 6.95\\
            \hline
            921 & 23.50 & 6.91\\
            \hline
            921 & 23.50 & 6.91\\
            \hline
            921 & 23.50 & 6.91\\
            \hline
            921 & 23.50 & 6.91\\
            \hline
            921 & 23.50 & 6.91\\
            \hline
            920 & 26.54 & 6.95\\
            \hline
            920 & 26.54 & 6.95\\
            \hline
            920 & 26.54 & 6.95\\
            \hline
            921 & 23.50 & 6.91\\
            \hline
            921 & 23.50 & 6.91\\
            \hline
      \end{tabular}
    }
    \caption{Misurazioni successive}
\end{table}
Il valore finale è stato calcolato come media su 20 misurazioni, da cui si è ricavato il valore $\overline{T}=(24.26\pm6.92)\SI{}{\celsius}$, con $D_{\text{out}, \max}=921$.
La misurazione è stata confrontata con quella di un secondo termometro, di incertezza supponibile trascurabile rispetto a quella riscontrata nell'esperienza, il quale ha riportato
una misura compatibile di $\SI{26.30}{\celsius}$. 
\section{Conclusioni}
L'esperienza ha portato a verificare come il PT100 assicuri una misura con incertezza minore rispetto al LM335 che abbiamo usato nella precedente esperienza e come utilizzare tecniche quali il dithering 
porti a imprecisioni minori nella misura finale.
\end{document}
