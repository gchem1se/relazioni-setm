\documentclass{article}
\usepackage[utf8]{inputenc}
\usepackage{listings}
\usepackage{xcolor}
\usepackage{siunitx}
\usepackage{amsmath}
\usepackage{float}
\usepackage{graphicx}
\usepackage[italian]{babel}
\definecolor{keyword}{rgb}{0,0,1}
\definecolor{codegray}{rgb}{0.5,0.5,0.5}
\definecolor{string}{rgb}{0,0.7,0.2}
\definecolor{backcolour}{rgb}{1,1,1}

\lstdefinestyle{mystyle}{
    backgroundcolor=\color{backcolour},   
    commentstyle=\color{codegray},
    keywordstyle=\color{keyword},
    numberstyle=\tiny\color{codegray},
    stringstyle=\color{string},
    basicstyle=\ttfamily\footnotesize,
    breakatwhitespace=false,         
    breaklines=true,                 
    captionpos=b,                    
    keepspaces=true,                 
    numbers=left,                    
    numbersep=5pt,                  
    showspaces=false,                
    showstringspaces=false,
    showtabs=false,                  
    tabsize=2
}

\lstset{style=mystyle}

\renewcommand*\contentsname{Indice}

\title{Misura di temperatura con
sensore PT100 e
scheda Arduino Uno}
\author{Simone Aronica, Giovanni Bloise, \\
Gabriele Camisa, Giuseppe Casale}

\begin{document}

\maketitle
\tableofcontents
\pagebreak

\section{TODO}
%\begin{itemize}
%\end{itemize}
\section{Strumenti usati}
%\begin{itemize}
%\end{itemize}

\section{Sintesi dell'esperienza}
L'esperienza consiste nella misurazione della temperatura ambientale in laboratorio mediante la scheda a microcontrollore Arduino Uno e un sensore di temperatura PT100. Bisogna anche riportare la correlazione di tali misure all'incertezza attesa e a valori di riferimento osservati su un secondo termometro digitale più accurato.

\section{Circuito di condizionamento}
La scheda Arduino Uno implementa un convertitore analogico-digitale a 10 bit con cui è possibile convertire un voltaggio in input, in un range da \SI{0}{\volt} alla tensione di riferimento di \SI{5}{\volt}, in codici digitali da 0 a 1023. 
Il sensore PT100 incorpora una resistenza dipendente dalla temperatura secondo: 
\begin{equation*}
    \begin{split}
        &R_{\theta}=R_0\cdot\left(1+A\cdot\theta+B\cdot\theta^2\right)\\
        &R_0=\SI{100}{\ohm}\\
        &A=3.9083\cdot10^{-3}\SI{}{\per\celsius}\\
        &B=-5.775\cdot10^{-7}\SI{}{\per\celsius}
    \end{split}
\end{equation*}
per cui è possibile ottenere come output la tensione su una resistenza posta in serie al sensore stesso.

Per ottimizzare la sensibilità $S_{V_F}^{\theta}=\frac{\partial{V_F}}{\partial{\theta}}$, rimanendo contemporaneamente in condizioni di autoriscaldamento accettabili, si è collegato il PT100 in una configurazione di partizione resistiva con un resistore di valore nominale $R_F=\SI{1}{\kilo\ohm}$ (successivamente rivalutato a $(976.00\pm10.76)\SI{}{\ohm}$ tramite il multimetro digitale). La tensione di output del circuito di condizionamento $V_F$ è stata prelevata ai capi di $R_F$.
La serie è stata alimentata tramite USB 2.0 ad una tensione di $V_S=(5.00\pm0.25)\SI{}{\volt}$.
Con questa configurazione si ha una sensibilità di $S_{V_F}^{\theta}=\SI{-1.54}{\milli\volt\per\celsius}$ e un autoriscaldamento $\Delta\theta_{\text{s-o}}\approx\SI{0.2}{\celsius}$.
\subsection{Valutazione dell'incertezza}
La funzione di taratura
\begin{equation*}
    \theta=-\frac{A}{2\cdot B} - \sqrt{\frac{A}{4\cdot B^2} - \frac{1}{R_0\cdot B}\cdot\left(R_0+R_F-\frac{V_S}{V_F}\cdot R_F\right)}
\end{equation*}
si può manipolare notando che è possibile eliminare la dipendenza da (TODO: Aggiungere Vcc):
\begin{equation*}
    \begin{split}
        V_q = \frac{V_{FR}}{2^{N_b}}=\frac{V_{CC}}{2^{N_b}}\\
        &D_{\text{out}} = V_{\text{CC}} \cdot \frac{R_F}{R_F + R_{\theta}}\cdot\frac{1}{V_{q}}\\
        &D_{\text{out}} = V_{\text{CC}} \cdot \frac{R_F}{R_F + R_{\theta}}\cdot\frac{2^{N_b}}{V_{CC}}\\
        &D_{\text{out}} = 5^{N_b} \cdot \frac{R_F}{R_F + R_{\theta}}
    \end{split}
\end{equation*}

L'incertezza assoluta sulla temperatura dipende quindi solo da $D_{\text{out}}$, $R_F$ e dal sensore.
\begin{equation*}
    \begin{split}
        \delta\theta&=\left|S_{\theta}^{V_s}\right|\delta V_s+\left|S_{\theta}^{V_F}\right|\delta V_F+\left|S_{\theta}^{R_F}\right|\delta R_F+\delta \theta^{\text{sens}}\\
        &=-\frac{R_f}{2B R_0 V_F \sqrt{\frac{A^2}{4B^2} - \frac{C-\frac{R_f V_s}{V_F}+R_F}{B R_0}}}\cdot\delta V_s +\\
        &\frac{R_F V_s}{2BR_0 V_F^2\sqrt{\frac{A^2}{4B^2}-\frac{R_0 - \frac{R_F V_s}{V_F} + R_F}{B R_0}}}\cdot\delta V_F +\\
        &\frac{1-\frac{V_S}{V_F}}{2B R_0 \sqrt{\frac{A^2}{4B^2} - \frac{R_0-\frac{R_F V_S}{V_F} +R_F}{B R_0}}}\cdot\delta R_F+\delta \theta^{\text{sens}}
    \end{split}
\end{equation*}
dove il termine $\delta\theta^\text{sens}=(0.3+0.005\cdot|\theta|)\SI{}{\celsius}$ incorpora le incertezze dovute ai parametri $A$, $B$ e $R_0$, dei quali non sono disponibili le singole incertezze.





\subsection{Firmware}
La funzione di taratura è stata implementata, insieme alla visualizzazione delle misure di temperatura nel seguente codice sorgente:
\begin{lstlisting}[language=C]
const int pin = A3;
const long Rf = 976;
const long R0 = 100;
const double A = 3.9083*pow(10, -3);
const double B = -5.775*pow(10, -7);
const int Vs = 5;

void setup() {
    pinMode(pin, INPUT);
    Serial.begin(9600);
}

void loop() {
    int Dout = analogRead(pin); 
    double T = -(A/(2*B))-sqrt(pow(A,2)/(4*pow(B,2))-1/(R0*B)*(R0+Rf-pow(2, 10)/Dout*Rf));
    
    Serial.print("Valore sensore: ");
    Serial.println(Dout);
    Serial.print("Valore temperatura in Celsius: ");
    Serial.println(T);  
    Serial.println("######");
    
    delay(1000);
}
\end{lstlisting}
\subsection{Dithering}
Il valore finale è calcolato come media su 20 misurazioni, da cui si ricava un valore di $\SI{26.85}{\celsius}$ e $D_{\text{out}, \max}=800$.
\section{Conclusioni}
\end{document}
