\documentclass{article}
\usepackage[utf8]{inputenc}
\usepackage{xcolor}
\usepackage{siunitx}
\usepackage{amsmath}
\usepackage{float}
\usepackage{graphicx}
\usepackage[italian]{babel}
\usepackage[normalem]{ulem}
\usepackage{circuitikz}

\ctikzset{voltage/bump b/.initial=0}

\renewcommand*\contentsname{Indice}

\title{Oscilloscopio digitale}
\author{Simone Aronica, Giovanni Bloise, \\
Gabriele Camisa, Giuseppe Casale}

\begin{document}

\maketitle
\tableofcontents
\pagebreak

\section*{TODO}

\section{Strumenti usati}
\begin{itemize}
    \item Multimetro HP 34401A
    \item Rigol DS1054-Z
    \item Generatore di funzioni
    \item Cavi coassiali
\end{itemize}

\section{Misurazione di valore efficace e frequenza}
La prima parte dell'esperienza consiste nella misurazione del valore efficace e della frequenza di un segnale sinusoidale
per mezzo dell'oscilloscopio digitale e nella correlazione dell'incertezza attesa. 

Il segnale misurato è stato generato tramite il generatore di funzioni come una sinusoide di ampiezza picco-picco \SI{1}{\volt} a frequenza \SI{1}{\kilo\hertz}.

\subsection{Misura del valore efficace}
La misura del valore efficace è stata derivata in maniera indiretta dalla misura dell'ampiezza picco-picco del segnale. 

Tale misura è stata effettuata tramite l'oscilloscopio digitale impostando la costante di taratura verticale a $K_{\text{V}} = \SI{200}{\milli\volt}/\text{div}$ e misurando i due valori di picco in negativo e in positivo del segnale 
(rispettivamente $y_1 = -2.5\ \text{div} $ e $y_2 = 2.5\ \text{div}$). È stata considerata un'incertezza di lettura pari a $\frac{1}{2}$ della risoluzione dello strumento (ovvero $\frac{1}{10}$ div).
che è stata valutata come 
$V_{\text{pp}}=(1.000\pm0.088)\SI{}{\volt}$. L'incertezza è stata calcolata secondo il modello deterministico:
\begin{equation*}
    \begin{split}
        \delta V_{\text{pp}}&=\left|\frac{\partial V_{\text{pp}}}{\partial y_1}\right|\delta y_1+\left|\frac{\partial V_{\text{pp}}}{\partial y_2}\right|\delta y_2+\left|\frac{\partial V_{\text{pp}}}{\partial K_{\text{V}}}\right|\delta K_{\text{V}}=\\
        &=\left|K_{\text{V}}\right|\delta y_1 + \left|K_{\text{V}}\right|\delta y_2 + (y_2 - y_1)\delta K_{\text{V}}
    \end{split}
\end{equation*}

\subsection{Misurazione di frequenza}

\subsection{Verifica con multimetro}

\section{Misurazione del tempo di salita}

\subsection{Operazioni preliminari}

\subsection{Tempo di salita in condizione di adattamento di impedenza}

\subsection{Tempo di salita con generatore ad alta impedenza (uso della sonda compensata)}



\section{Conclusioni}

\end{document}
